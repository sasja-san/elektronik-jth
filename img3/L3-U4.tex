
\begin{circuitikz}[scale=1.0]
  \tikzset{
    mux 8by3 wl/.style={muxdemux, muxdemux def={Lh=10, NL=10, Rh=3, NB=5, w=4},
    draw only left pins={2-9},
    draw only bottom pins={2-4},
    muxdemux label={
      B2=\ix{A}{1},
      B3=\ix{A}{2},
      B4=\ix{A}{4},
      L2=0,
      L3=1,
      L4=2,
      L5=3,
      L6=4,
      L7=5,
      L8=6,
      L9=7,
    },   
    circuitikz/muxdemux/inner label ysep=4pt}
  }
  \node
    [mux 8by3 wl](M){}
  ;
  \draw
    (M.bpin 2) -- ++ (0,-0.5) coordinate (A) node[below]{C}
    (M.bpin 3) -- (M.bpin 3 |- A) node[below]{B}
    (M.bpin 4) -- (M.bpin 4 |- A) node[below]{A}
    
    (M.lpin 2) -- ++(-0.5,0) node[left]{1}
    (M.lpin 3) -- ++(-0.5,0) node[left]{1}
    (M.lpin 4) -- ++(-0.5,0) node[left]{0}
    (M.lpin 5) -- ++(-0.5,0) node[left]{1}
    (M.lpin 6) -- ++(-0.5,0) node[left]{0}
    (M.lpin 7) -- ++(-0.5,0) node[left]{1}
    (M.lpin 8) -- ++(-0.5,0) node[left]{0}
    (M.lpin 9) -- ++(-0.5,0) node[left]{1}

    (M.rpin 1) -- ++(0.5,0) node[right]{X}
  ;
\end{circuitikz}
